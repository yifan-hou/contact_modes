\documentclass[conference]{IEEEtran}
\usepackage{times}

% numbers option provides compact numerical references in the text.
\usepackage[numbers]{natbib}
\usepackage{multicol}
\usepackage[bookmarks=true]{hyperref}
\usepackage{amsmath,amssymb}
\newcommand{\Mod}[1]{\ (\mathrm{mod}\ #1)}
% \usepackage[standard]{ntheorem}
\usepackage{amsthm}
\usepackage{mathtools}
\usepackage{bm}
\usepackage{graphicx}
% \usepackage{caption}
% \usepackage{figure}
\usepackage{float}
\usepackage{subcaption}
\usepackage{epstopdf}
\usepackage{dblfloatfix}
\usepackage{fixltx2e}
% \usepackage{subfig}
\usepackage{mathrsfs}
\usepackage{algorithm}
\usepackage{algorithmicx}
\usepackage[noend]{algpseudocode}
\algnewcommand\algorithmicC{\textbf{continue}}
\algnewcommand\Continue{\algorithmicC}
\algnewcommand\algorithmicB{\textbf{break}}
\algnewcommand\Break{\algorithmicB}
\makeatletter
\def\BState{\State\hskip-\ALG@thistlm}

\def\subsubsection{\@startsection{subsubsection}% name
                                 {3}% level
                                 {\z@}% indent (formerly \parindent)
                                 {0.5ex plus .5ex minus 0ex}% before skip
                                 {0.5ex plus .5ex minus 0ex}% after skip
                                 {\normalfont\normalsize\itshape}}% style

% \def\ALG@step%
%   {%
%   \refstepcounter{ALG@line}% Step and anchor for hyperref
%   \stepcounter{ALG@rem}% Regular step (equivalent to \addtocounter{ALG@rem}{1})
%   \ifthenelse{\equal{\arabic{ALG@rem}}{\ALG@numberfreq}}%
%     {\setcounter{ALG@rem}{0}\alglinenumber{\arabic{ALG@line}}}%
%     {}%
%   }%

\makeatother
\algdef{SE}[DOWHILE]{Do}{DoWhile}[1]{\algorithmicdo\ #1}[1]{\algorithmicwhile\ #1}

\usepackage{booktabs}
\newcommand\Tstrut{\rule{0pt}{2.5ex}}       % "top" strut
\newcommand\Bstrut{\rule[-0.9ex]{0pt}{0pt}} % "bottom" strut
\newcommand{\TBstrut}{\Tstrut\Bstrut} % top&bottom struts

\usepackage{tabstackengine}
\stackMath

\usepackage{tikz}
\usetikzlibrary{scopes}
\usetikzlibrary{shapes.misc}
\tikzset{cross/.style={cross out, draw=black, minimum size=2*(#1-\pgflinewidth), inner sep=0pt, outer sep=0pt},
%default radius will be 1pt.
cross/.default={2pt}}

\let\labelindent\relax
\usepackage{enumitem}
% \newenvironment{enum}{\begin{enumerate}[wide, labelwidth=!, labelindent=0pt]}{\end{enumerate}}
\newlist{inparaenum}{enumerate}{2}% allow two levels of nesting in an enumerate-like environment
\setlist[inparaenum]{nosep,wide,labelwidth=!,labelindent=0pt}% compact spacing for all nesting levels
\setlist[inparaenum,1]{label=\bfseries\arabic*)}% labels for top level
\setlist[inparaenum,2]{label=\arabic{inparaenumi}{\alph*})}% labels for second level


\newtheorem{theorem}{Theorem}
\newtheorem{proposition}{Proposition}
\newtheorem{definition}{Definition}
\newtheorem{corollary}{Corollary}
\newcommand\numberthis{\addtocounter{equation}{1}\tag{\theequation}}
\DeclareMathOperator{\sign}{\text{sgn}}
\DeclareMathOperator*{\argmin}{arg\,min}
\DeclareMathOperator{\intr}{int}
\DeclareMathOperator{\dom}{dom}
\DeclareMathOperator{\rot}{\text{rot}}
\DeclareMathOperator{\adjoint}{Ad}

\newcommand{\TODO}[1]{{\color{red} {{#1}}  }}

\pdfinfo{
   /Author (Eric Huang; Xianyi Chang; Matthew T. Mason)
   /Title  (Efficient Contact Mode Enumeration in 2D and 3D)
   /CreationDate (D:20161016120000)
   /Subject (Robots)
   /Keywords (Manipulation)
}

\begin{document}

% paper title
\title{\huge Efficient Contact Mode Enumeration in 2D and 3D}

% You will get a Paper-ID when submitting a pdf file to the conference system
\author{Author Names Omitted for Anonymous Review. Paper-ID [?]}

% \author{\authorblockN{Eric Huang and Matthew T. Mason}
% \authorblockA{Robotics Institute\\
% Carnegie Mellon University,
% Pittsburgh, Pennsylvania 15213\\ erich1@andrew.cmu.edu, matt.mason@cs.cmu.edu}}

% avoiding spaces at the end of the author lines is not a problem with
% conference papers because we don't use \thanks or \IEEEmembership

\maketitle

\begin{abstract}

\end{abstract}
 
\IEEEpeerreviewmaketitle

\section{Introduction}

\section{Related Work}

\begin{inparaenum}
    \item Polyhedral convex cones
    \item Matt's book
    \item Max Haas's work
    \item Bouligand derivative
\end{inparaenum}

\section{Background}

The normal velocity equation
\begin{equation}
B^T\adjoint_{g_{oc}}^{-1}\xi = 
B^T \begin{bmatrix}
        R_{oc}^T & -R_{oc}^T\widehat{p}\\
        0 & R^T
    \end{bmatrix}\xi = 
\begin{bmatrix} n & -n\widehat{p} \end{bmatrix}\xi
\end{equation}

\section{Contact Mode Enumeration in 2D}

\begin{theorem}
    The proposed algorithm is order $O(n\log n)$. Moreover the number of
    distinct contact modes is order $O(n)$.
\end{theorem}

\begin{proof}
    todo
\end{proof}

\section{Contact Mode Enumeration in 3D}

\section{Implementation}

\subsection{Hardware \& Software}
All algorithms were run on a computer with an Intel i7-7820x CPU (3.5 MHz, 16
threads). Implementations of the algorithms described in this paper are freely
available at \url{http://www.github.com/<omitted>/contact_modes}.

\section{Results}

\subsection{Runtimes for 2D Problems}

\subsection{Runtimes for 3D Problems}

\section{Conclusion}

\bibliographystyle{plainnat}
\bibliography{references}

\end{document}


