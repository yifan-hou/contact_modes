\documentclass[conference]{IEEEtran}
\usepackage{times}

% numbers option provides compact numerical references in the text.
\usepackage[numbers]{natbib}
\usepackage{multicol}
\usepackage[bookmarks=true]{hyperref}
\usepackage{amsmath,amssymb}
\newcommand{\Mod}[1]{\ (\mathrm{mod}\ #1)}
% \usepackage[standard]{ntheorem}
\usepackage{amsthm}
\usepackage{mathtools}
\usepackage{bm}
\usepackage{graphicx}
% \usepackage{caption}
% \usepackage{figure}
\usepackage{float}
\usepackage{subcaption}
\usepackage{epstopdf}
\usepackage{dblfloatfix}
\usepackage{fixltx2e}
% \usepackage{subfig}
\usepackage{mathrsfs}
\usepackage{algorithm}
\usepackage{algorithmicx}
\usepackage[noend]{algpseudocode}
\algnewcommand\algorithmicC{\textbf{continue}}
\algnewcommand\Continue{\algorithmicC}
\algnewcommand\algorithmicB{\textbf{break}}
\algnewcommand\Break{\algorithmicB}
\algnewcommand{\LineComment}[1]{\State {\color{blue}\(\triangleright\) #1}}
\makeatletter
\def\BState{\State\hskip-\ALG@thistlm}

\def\subsubsection{\@startsection{subsubsection}% name
                                 {3}% level
                                 {\z@}% indent (formerly \parindent)
                                 {0.5ex plus .5ex minus 0ex}% before skip
                                 {0.5ex plus .5ex minus 0ex}% after skip
                                 {\normalfont\normalsize\itshape}}% style

% \def\ALG@step%
%   {%
%   \refstepcounter{ALG@line}% Step and anchor for hyperref
%   \stepcounter{ALG@rem}% Regular step (equivalent to \addtocounter{ALG@rem}{1})
%   \ifthenelse{\equal{\arabic{ALG@rem}}{\ALG@numberfreq}}%
%     {\setcounter{ALG@rem}{0}\alglinenumber{\arabic{ALG@line}}}%
%     {}%
%   }%

\makeatother
\algdef{SE}[DOWHILE]{Do}{DoWhile}[1]{\algorithmicdo\ #1}[1]{\algorithmicwhile\ #1}

\usepackage{booktabs}
\newcommand\Tstrut{\rule{0pt}{2.5ex}}       % "top" strut
\newcommand\Bstrut{\rule[-0.9ex]{0pt}{0pt}} % "bottom" strut
\newcommand{\TBstrut}{\Tstrut\Bstrut} % top&bottom struts

\usepackage{tabstackengine}
\stackMath

\usepackage{tikz}
\usetikzlibrary{scopes}
\usetikzlibrary{shapes.misc}
\tikzset{cross/.style={cross out, draw=black, minimum size=2*(#1-\pgflinewidth), inner sep=0pt, outer sep=0pt},
%default radius will be 1pt.
cross/.default={2pt}}

\let\labelindent\relax
\usepackage{enumitem}
% \newenvironment{enum}{\begin{enumerate}[wide, labelwidth=!, labelindent=0pt]}{\end{enumerate}}
\newlist{inparaenum}{enumerate}{2}% allow two levels of nesting in an enumerate-like environment
\setlist[inparaenum]{nosep,wide,labelwidth=!,labelindent=0pt}% compact spacing for all nesting levels
\setlist[inparaenum,1]{label=\bfseries\arabic*)}% labels for top level
\setlist[inparaenum,2]{label=\arabic{inparaenumi}{\alph*})}% labels for second level


\newtheorem{theorem}{Theorem}
\newtheorem{proposition}{Proposition}
\newtheorem{definition}{Definition}
\newtheorem{corollary}{Corollary}
\newcommand\numberthis{\addtocounter{equation}{1}\tag{\theequation}}
\DeclareMathOperator{\sign}{\text{sgn}}
\DeclareMathOperator*{\argmin}{arg\,min}
\DeclareMathOperator{\intr}{int}
\DeclareMathOperator{\dom}{dom}
\DeclareMathOperator{\rot}{\text{rot}}
\DeclareMathOperator{\adjoint}{Ad}
\DeclareMathOperator{\relint}{relint}
\DeclareMathOperator{\aff}{aff}

\newcommand{\TODO}[1]{{\color{red} {{#1}}  }}

\pdfinfo{
   /Author (Eric Huang; Xianyi Chang; Matthew T. Mason)
   /Title  (Efficient Contact Mode Enumeration in 2D and 3D)
   /CreationDate (D:20161016120000)
   /Subject (Robots)
   /Keywords (Manipulation)
}

\begin{document}

% paper title
\title{\huge Efficient Contact Mode Enumeration in 2D and 3D}

% You will get a Paper-ID when submitting a pdf file to the conference system
\author{Author Names Omitted for Anonymous Review. Paper-ID [?]}

% \author{\authorblockN{Eric Huang and Matthew T. Mason}
% \authorblockA{Robotics Institute\\
% Carnegie Mellon University,
% Pittsburgh, Pennsylvania 15213\\ erich1@andrew.cmu.edu, matt.mason@cs.cmu.edu}}

% avoiding spaces at the end of the author lines is not a problem with
% conference papers because we don't use \thanks or \IEEEmembership

\maketitle

\begin{abstract}
This paper investigates the problem of polynomial-time contact mode enumeration
for use within contact mechanics.
\end{abstract}
 
\IEEEpeerreviewmaketitle

\section{Introduction}

\begin{inparaenum}
    \item Detecting sliding slipping modes
    \item Contact modes dictate the contact dynamics 
\end{inparaenum}

\section{Related Work}

\begin{inparaenum}
    \item \TODO{Xianyi} Polyhedral convex cones
    \item \citet{mason_mechanics_2001} sketched an algorithm for contact mode
    enumeration in 2D which intersects the positive (negative) rotation centers
    on the positive (negative) oriented plane and intersects the rotation
    centers at infinity on the equator. Though \citet{mason_mechanics_2001}
    upper-bounded the number of modes at $O(n^2)$, the algorithm's runtime is
    actually $O(n\log n)$ and the correct number of modes is $\Theta(n)$.
    Unfortunately, the oriented plane technique does not generalize to 3D
    contact mode enumeration.
    \item \citet{haas-heger_passive_2018} independently published an algorithm
    for partial contact mode enumeration in 2D. There, they interpret the
    feasible modes as the regions of an arrangement of hyperplanes in 3D.
    However, \citet{haas-heger_passive_2018}'s algorithm is at least
    $\Omega(n^4)$ and does not enumerate separating modes. Regardless of these
    issues, their work inspired us to investigate hyperplane arrangements in
    higher dimensions for our algorithm.
    \item To the best of our knowledge, our algorithm is the first known method
    for contact mode enumeration in 3D.
    \item Nikhil's work
    \item Impacts for grasping
    \item Bouligand derivative
\end{inparaenum}

\section{Mechanics of Contact}

\TODO{}

The normal velocity equation
\begin{equation}
B^T\adjoint_{g_{oc}}^{-1}\xi = 
B^T \begin{bmatrix}
        R_{oc}^T & -R_{oc}^T\widehat{p}\\
        0 & R^T
    \end{bmatrix}\xi = 
\begin{bmatrix} n & -n\widehat{p} \end{bmatrix}\xi
\end{equation}

\section{Convex Polytopes}

\TODO{Xianyi, Eric give bullet points}

Let $P \subseteq \mathbb{R}^d$ be a convex set. This work primarily uses the
following two classes of convex sets. The $\mathcal{H}\text{-}polyhedron$ is an
intersection of closed halfspaces
\begin{equation}
    \mathcal{H}(A,z) = \{x \in \mathbb{R}^d : Ax \leq z\}.
\end{equation}
The $\mathcal{V}\text{-}polytope$ is the convex hull of a finite point set
\begin{equation}
    \mathcal{V}(A) = \{x \in \mathbb{R}^d : x = At, t \geq 0, \mathbf{1}t = 1\}.
\end{equation}

Let $P \subseteq \mathbb{R}^d$ be a convex polyhedron. Let a \textit{face} of
$P$ be any set of the form 
\begin{equation}
    F = P \cap \{x : cx = c_0, x \in \mathbb{R}^d\}.
\end{equation}
The dimension of a face is the dimension of its affine hull $\dim(F) =
\dim(\aff(F))$. The faces of dimensions $0$, $1$, $\dim(P)-2$, and $\dim(P)-1$
are called \textit{vertices}, \textit{edges}, \textit{ridges}, and
\textit{facets}, respectively.

\subsection{Face Lattice}
The face lattice $L(P)$ is the partially ordered set (poset) of all faces of a polytope $P$, partially ordered by inclusion. Figure.\ref{} visualize the face lattice of a cube. The minimal element at the bottom is the empty face. The first layer of nodes correspond to eight vertices of the cube. The second layer of nodes represents the edges, while the nodes of the third layers are the six facets of the cube. The top element is the cube itself. 

(Write this as theorem) There are several properties of the face lattice $L(P)$ of a polytope $P$ that are used in this paper:
\begin{enumerate}
    \item Every maximum chain of $L(P)$ has the length of $dim(P) + 1$. 
    \item The reversed order of $L(P)$ is the opposite $L(P)^{op}$, which is also the face lattice of a convex polytope.
\end{enumerate}


\subsection{Polarity}
Here we define the polar(dual) polytope $P^*$ of a polytope $P \subseteq \mathbb{R}^d$ 
\begin{equation}
    P^* = \{c \in \mathbb{R}^d: c^T x \leq 1, \forall x \in P\} \subseteq \mathbb{R}^d
\end{equation}
In this definition, we assume that $\mathbf{0}$ is in the interior of the polytope $P$ without loss of generality.   

The face lattice of the polar polytope $P^*$ is the opposite of the face lattice of $P$:
\begin{equation}
    L(P^*) = L(P)^{op}
\end{equation}
From this theorem, we get the translation the inclusion of faces and interchanges:
\begin{align}
    \emptyset &\longleftrightarrow P \\
    vertices &\longleftrightarrow facets \\
    edges &\longleftrightarrow ridges \\
    ... &\longleftrightarrow ...
\end{align}

\subsection{Fans}
A fan in $\mathbb{R}^d$ is a family 
\begin{equation}
    \mathcal{F} = \{C_1, C_2, \hdots, C_N\}
\end{equation}
of nonempty polyhedral cones, with the following two properties:
\begin{enumerate}
    \item Every nonempty face of a cone in $\mathcal{F}$ is also a cone in $\mathcal{F}$.
    \item The intersection of any two cones $in \mathcal{F}$ is a face of both.
\end{enumerate}
The fan $\mathcal{F}$ is \textit{complete} if the $\cup_{i=1}^N C_i = \mathbb{R}^d$.

The arrangement/lattice of a Fan

\section{Contact Mode Enumeration in 2D}

For completeness' sake, this section provides an updated bound on the 2D contact
mode enumeration algorithm outlined in \citet{Mason}. The algorithm was
published with no analysis of the runtime and an $O(n^2)$ bound on the number of
contact modes. We show that the runtime is optimal at $O(n\log n)$ and argue
that the number of modes is in fact $O(n)$. 

\begin{theorem}
    The proposed algorithm is order $O(n\log n)$. Moreover the number of
    distinct contact modes is order $O(n)$.
\end{theorem}

\begin{proof}
    todo
\end{proof}

\section{Contact Mode Enumeration in 3D}

\subsection{Contacting/Separating Mode Enumeration}

\begin{algorithm}[t]
    \caption{C/S Mode Enumeration}\label{alg:match}
    \begin{algorithmic}[1]
        \Function{CS-Enumerate}{$P$, $N$}
        \LineComment{\text{Build normal velocity constraint matrix}}
        % \State $A \gets \begin{bmatrix}n_0 & -n_0\widehat{p}_0\\\vdots & \vdots\\ n_k & -n_k\widehat{p}_k\end{bmatrix},\; n_i = N_{\{i,\cdot\}},\; p_i = P_{\{i,\cdot\}}$
        \State $A_{\{i,\cdot\}} \gets \begin{bmatrix}n_i & -n_i\widehat{p}_i\end{bmatrix},\; n_i = N_{\{i,\cdot\}},\; p_i = P_{\{i,\cdot\}}$
        % \State $P \gets \mathcal{H}(A,0)$
        \LineComment{\text{\TODO{TODO: Project to DIM(AFF(A))}}}
        \LineComment{\text{Convert to polar form}}
        \State $r \gets \Call{RelInt-Point}{\mathcal{H}(A,0)}$
        \State $z \gets -Ar$
        \State $A_{\{i,\cdot\}} \gets A_{\{i,\cdot\}} / z_i$
        % \State $\mathcal{H}(A',z) \gets \Call{Translate}{\mathcal{H}(A,0), r}$
        % \State $\mathcal{V}(A'') \gets \Call{Polar}{\mathcal{H}(A',z)}$
        \LineComment{\text{Get facets from polar convex hull}}
        \State $M \gets \Call{Conv-Hull}{\mathcal{V}(A^T)}$
        \LineComment{\text{Build face lattice}}
        \State $d \gets \Call{Dim-Aff}{A}$
        \State $n_v, n_f \gets \Call{Size}{M}$
        \State $L \gets \Call{List}{\varnothing, d+1}$
        \State $L[0] \gets \{\Call{Range}{n_v}\}$
        \For {$i \in \{0, \ldots, n_f-1\}$}
        \State $L[1] \gets L[1] \cup \{\Call{Where}{M[:,i]}\}$
        \EndFor
        \For {$i \in \{1, \ldots, d-2\}$}
        \For {$F, G \in L[i],\, F \neq G$}
        \State $H \gets F \cap G$
        \If {$\Call{Len}{H} \geq d-i-2$}
        \State $L[i+1] \gets L[i+1]\cup\{H\}$
        \EndIf
        \EndFor
        \EndFor
        \State $L[d]\gets \{\varnothing\}$
        \LineComment{\text{Convert faces to mode strings}}
        \State $S \gets \varnothing$
        \For {$i \in \{0, \ldots, d\}$}
        \For {$F \in L[i]$}
        \State $m \gets [s]\times n_v$
        \State $m[F] \gets c$
        \State $S \gets S \cup \{m\}$
        \EndFor
        \EndFor
        \State \Return $S$
        \EndFunction
    \end{algorithmic}
\end{algorithm}

\TODO{Eric}

We present our 

\subsubsection{Build normal velocity constraint matrix} 

\subsubsection{Project to DIM(AFF(A))} 

\subsubsection{Convert to polar form} 

The polar form $P^\Delta$ of a polyhedra $P$ is defined only when $0 \in
\relint(P)$. However, $0$ is on the boundary of the polyhedral cone
$\mathcal{H}(A, 0)$. Therefore, our first step is to find a point $r \in
\relint(\mathcal{H}(A,0))$. This is a classical problem in linear programming,
and for our implementation, it amounts to solving the following linear program

\begin{alignat}{2}
    &\!\!\min_{x} && {\begin{bmatrix}0& 1\end{bmatrix} \cdot
    \begin{bmatrix}x\\c\end{bmatrix}} \label{eq:itm-obj}\\
    &\textup{s.t.} & & 
    \begin{bmatrix}A & 1 \\ B & 1 \end{bmatrix} \begin{bmatrix} x \\c
    \end{bmatrix} \leq 0\label{eq:itm-eq}\\
    & & 
    0\leq &
    \begin{bmatrix}
    \alpha \\ \beta \\ \gamma
    \end{bmatrix}
    \bot
    \begin{bmatrix}
    \lambda_n \\ \lambda_f \\ \sigma
    \end{bmatrix}
    \geq 0. \label{eq:itm-ineq}
\end{alignat}

\subsubsection{Build face lattice} 

\subsubsection{Convert faces to mode strings} 

\subsection{Sticking/Sliding Mode Enumeration}

\TODO{Xianyi}

\begin{algorithm}[t]
    \caption{S/S Mode Enumeration}\label{alg:match}
    \begin{algorithmic}[1]
        \Function{SS-Enumerate}{$P$, $N$, $m$}
        \State $p_1, p_2, p_3 \gets \Call{Random-Select-3}{points}$
        \State $\ell_1, \ell_2, \ell_3 \gets \Call{Distances}{p_1,p_2,p_3}$
        \State $F_{1,2} \gets \emptyset$
        \For {$f_1 \in mesh.faces$}
        \State $F_2 \gets \Call{Query}{segtrees[f_1], \ell_1}$
        \State $F_{1,2} \gets F_{1,2} \cup \{f_1,f_2\}, \, \forall f_2 \in F_2$
        \EndFor
        \State $F_{1,2,3} \gets \emptyset$
        \For {$\{f_1, f_2\} \in F_{1,2}$}
        \State $F_3 \gets \Call{Query}{segtrees[f_2], \ell_2}$
        \For {$f_3 \in F_3$}
        \If {$\ell_3 \in \Call{Range}{f_1,f_3}$}
        \State $F_{1,2,3} \gets F_{1,2,3} \cup \{f_1,f_2,f_3\}$
        \EndIf
        \EndFor
        \EndFor
        \State $T_p \gets \Call{Compute-Transform}{p_1,p_2,p_3}$
        \State $matches \gets []$
        \For {$\{f_1, f_2, f_3\} \in F_{1,2,3}$}
        \State $T_f \gets \Call{Compute-Transform}{f_1,f_2,f_3}$
        \State $p_f \gets T_fT_p^{-1}\cdot points$
        \State $score \gets \sum_{p \in p_f} sdf[p]$
        \State $matches \gets matches \cup \{score, T_f\}$
        \EndFor
        \State $matches \gets \Call{Sort}{matches}$
        \State \Return $matches[0:k]$
        \EndFunction
    \end{algorithmic}
\end{algorithm}

\section{Implementation}

\subsection{Hardware \& Software}
All algorithms were run on a computer with an Intel i7-7820x CPU (3.5 MHz, 16
threads). Implementations of the algorithms described in this paper are freely
available at \url{http://www.github.com/<omitted>/contact_modes}.

\section{Results}

\subsection{Runtimes for 2D Problems}

\subsection{Runtimes for 3D Problems}

\section{Experiments}

\subsection{Visualizing \& Benchmarking}

\subsection{Simulation (?)}

\subsection{Optimization (?)}

\subsection{Planning (?)}

\section{Conclusion}

\bibliographystyle{plainnat}
% \bibliography{references}
\bibliography{zotero}

\end{document}


